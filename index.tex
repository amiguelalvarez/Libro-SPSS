% Options for packages loaded elsewhere
\PassOptionsToPackage{unicode}{hyperref}
\PassOptionsToPackage{hyphens}{url}
\PassOptionsToPackage{dvipsnames,svgnames,x11names}{xcolor}
%
\documentclass[
  letterpaper,
  DIV=11,
  numbers=noendperiod]{scrreprt}

\usepackage{amsmath,amssymb}
\usepackage{iftex}
\ifPDFTeX
  \usepackage[T1]{fontenc}
  \usepackage[utf8]{inputenc}
  \usepackage{textcomp} % provide euro and other symbols
\else % if luatex or xetex
  \usepackage{unicode-math}
  \defaultfontfeatures{Scale=MatchLowercase}
  \defaultfontfeatures[\rmfamily]{Ligatures=TeX,Scale=1}
\fi
\usepackage{lmodern}
\ifPDFTeX\else  
    % xetex/luatex font selection
\fi
% Use upquote if available, for straight quotes in verbatim environments
\IfFileExists{upquote.sty}{\usepackage{upquote}}{}
\IfFileExists{microtype.sty}{% use microtype if available
  \usepackage[]{microtype}
  \UseMicrotypeSet[protrusion]{basicmath} % disable protrusion for tt fonts
}{}
\makeatletter
\@ifundefined{KOMAClassName}{% if non-KOMA class
  \IfFileExists{parskip.sty}{%
    \usepackage{parskip}
  }{% else
    \setlength{\parindent}{0pt}
    \setlength{\parskip}{6pt plus 2pt minus 1pt}}
}{% if KOMA class
  \KOMAoptions{parskip=half}}
\makeatother
\usepackage{xcolor}
\setlength{\emergencystretch}{3em} % prevent overfull lines
\setcounter{secnumdepth}{5}
% Make \paragraph and \subparagraph free-standing
\makeatletter
\ifx\paragraph\undefined\else
  \let\oldparagraph\paragraph
  \renewcommand{\paragraph}{
    \@ifstar
      \xxxParagraphStar
      \xxxParagraphNoStar
  }
  \newcommand{\xxxParagraphStar}[1]{\oldparagraph*{#1}\mbox{}}
  \newcommand{\xxxParagraphNoStar}[1]{\oldparagraph{#1}\mbox{}}
\fi
\ifx\subparagraph\undefined\else
  \let\oldsubparagraph\subparagraph
  \renewcommand{\subparagraph}{
    \@ifstar
      \xxxSubParagraphStar
      \xxxSubParagraphNoStar
  }
  \newcommand{\xxxSubParagraphStar}[1]{\oldsubparagraph*{#1}\mbox{}}
  \newcommand{\xxxSubParagraphNoStar}[1]{\oldsubparagraph{#1}\mbox{}}
\fi
\makeatother


\providecommand{\tightlist}{%
  \setlength{\itemsep}{0pt}\setlength{\parskip}{0pt}}\usepackage{longtable,booktabs,array}
\usepackage{calc} % for calculating minipage widths
% Correct order of tables after \paragraph or \subparagraph
\usepackage{etoolbox}
\makeatletter
\patchcmd\longtable{\par}{\if@noskipsec\mbox{}\fi\par}{}{}
\makeatother
% Allow footnotes in longtable head/foot
\IfFileExists{footnotehyper.sty}{\usepackage{footnotehyper}}{\usepackage{footnote}}
\makesavenoteenv{longtable}
\usepackage{graphicx}
\makeatletter
\def\maxwidth{\ifdim\Gin@nat@width>\linewidth\linewidth\else\Gin@nat@width\fi}
\def\maxheight{\ifdim\Gin@nat@height>\textheight\textheight\else\Gin@nat@height\fi}
\makeatother
% Scale images if necessary, so that they will not overflow the page
% margins by default, and it is still possible to overwrite the defaults
% using explicit options in \includegraphics[width, height, ...]{}
\setkeys{Gin}{width=\maxwidth,height=\maxheight,keepaspectratio}
% Set default figure placement to htbp
\makeatletter
\def\fps@figure{htbp}
\makeatother

\KOMAoption{captions}{tableheading}
\makeatletter
\@ifpackageloaded{bookmark}{}{\usepackage{bookmark}}
\makeatother
\makeatletter
\@ifpackageloaded{caption}{}{\usepackage{caption}}
\AtBeginDocument{%
\ifdefined\contentsname
  \renewcommand*\contentsname{Tabla de contenidos}
\else
  \newcommand\contentsname{Tabla de contenidos}
\fi
\ifdefined\listfigurename
  \renewcommand*\listfigurename{Listado de Figuras}
\else
  \newcommand\listfigurename{Listado de Figuras}
\fi
\ifdefined\listtablename
  \renewcommand*\listtablename{Listado de Tablas}
\else
  \newcommand\listtablename{Listado de Tablas}
\fi
\ifdefined\figurename
  \renewcommand*\figurename{Figura}
\else
  \newcommand\figurename{Figura}
\fi
\ifdefined\tablename
  \renewcommand*\tablename{Tabla}
\else
  \newcommand\tablename{Tabla}
\fi
}
\@ifpackageloaded{float}{}{\usepackage{float}}
\floatstyle{ruled}
\@ifundefined{c@chapter}{\newfloat{codelisting}{h}{lop}}{\newfloat{codelisting}{h}{lop}[chapter]}
\floatname{codelisting}{Listado}
\newcommand*\listoflistings{\listof{codelisting}{Listado de Listados}}
\makeatother
\makeatletter
\makeatother
\makeatletter
\@ifpackageloaded{caption}{}{\usepackage{caption}}
\@ifpackageloaded{subcaption}{}{\usepackage{subcaption}}
\makeatother

\ifLuaTeX
\usepackage[bidi=basic]{babel}
\else
\usepackage[bidi=default]{babel}
\fi
\babelprovide[main,import]{spanish}
% get rid of language-specific shorthands (see #6817):
\let\LanguageShortHands\languageshorthands
\def\languageshorthands#1{}
\ifLuaTeX
  \usepackage{selnolig}  % disable illegal ligatures
\fi
\usepackage{bookmark}

\IfFileExists{xurl.sty}{\usepackage{xurl}}{} % add URL line breaks if available
\urlstyle{same} % disable monospaced font for URLs
\hypersetup{
  pdftitle={Análisis de Datos con SPSS},
  pdfauthor={Angel Miguel Alvarez Calicho},
  pdflang={es},
  colorlinks=true,
  linkcolor={blue},
  filecolor={Maroon},
  citecolor={Blue},
  urlcolor={Blue},
  pdfcreator={LaTeX via pandoc}}


\title{Análisis de Datos con SPSS}
\author{Angel Miguel Alvarez Calicho}
\date{2024-12-23}

\begin{document}
\maketitle

\renewcommand*\contentsname{Tabla de contenidos}
{
\hypersetup{linkcolor=}
\setcounter{tocdepth}{2}
\tableofcontents
}

\bookmarksetup{startatroot}

\chapter*{Bienvenida}\label{bienvenida}
\addcontentsline{toc}{chapter}{Bienvenida}

\markboth{Bienvenida}{Bienvenida}

¡Bienvenido! Este proyecto se dedica a explorar las diversas
aplicaciones de SPSS en el análisis estadístico, a partir de mi
experiencia y estudios en esta área. A lo largo de este libro,
compartiremos conocimientos y técnicas que te permitirán utilizar SPSS
de manera efectiva.

Mi nombre es Angel Miguel Alvarez Calicho, soy graduado en la
Licenciatura en Ingeniería Comercial y tengo una Maestría en Estadística
Aplicada a la Gestión Empresarial. Mi pasión por la estadística y el
análisis de datos me ha llevado a impartir clases sobre SPSS, lo que me
ha permitido identificar las necesidades y desafíos que enfrentan los
estudiantes y profesionales al utilizar este software.

\includegraphics[width=1\textwidth,height=\textheight]{images/Portada Libro de SPSS.webp}

\section*{¿Por qué este libro?}\label{por-quuxe9-este-libro}
\addcontentsline{toc}{section}{¿Por qué este libro?}

\markright{¿Por qué este libro?}

A menudo he observado que muchos de mis alumnos y colegas tienen la
necesidad de contar con un manual práctico y accesible de SPSS. Aunque
existen numerosos recursos, muchos de ellos están desactualizados o
presentan versiones de interfaz muy antiguas, como la versión 12 o
anteriores. Además, los libros de pago, que son una opción válida, no
son fácilmente accesibles en Bolivia. Por esta razón, he decidido crear
este manual, con el objetivo de proporcionar una guía completa y
actualizada para aprender y recordar el uso de SPSS.

\section*{¿A quién va dirigido?}\label{a-quiuxe9n-va-dirigido}
\addcontentsline{toc}{section}{¿A quién va dirigido?}

\markright{¿A quién va dirigido?}

Este libro está destinado a cualquier persona que desee aprender a
utilizar SPSS, ya sea desde un nivel básico hasta uno avanzado. Si eres
estudiante, profesional o simplemente tienes interés en el análisis de
datos, aquí encontrarás un recurso útil que te guiará paso a paso en el
uso de esta herramienta.

\section*{Agradecimientos}\label{agradecimientos}
\addcontentsline{toc}{section}{Agradecimientos}

\markright{Agradecimientos}

Quiero expresar mi sincero agradecimiento a mis docentes, especialmente
a Sergio Rivero, quien me enseñó a apreciar el valor de SPSS en la
investigación de mercados. También agradezco al Licenciado Mauricio
Alarcón por brindarme la oportunidad de dictar talleres de SPSS, lo que
me permitió profundizar en el uso de esta herramienta y compartir mis
conocimientos con otros.

\section*{El Software SPSS}\label{el-software-spss}
\addcontentsline{toc}{section}{El Software SPSS}

\markright{El Software SPSS}

SPSS, que significa Statistical Package for the Social Sciences, es una
herramienta de análisis estadístico ampliamente utilizada en diversas
disciplinas. Actualmente, la versión más reciente es la 30, al momento
de publicar esta actualización el 31 de octubre de 2024, pero en estos
tutoriales veremos la versión 27. Este software es conocido por su
interfaz intuitiva y su capacidad para manejar grandes volúmenes de
datos de manera eficiente.

\subsection*{¿Por qué SPSS?}\label{por-quuxe9-spss}
\addcontentsline{toc}{subsection}{¿Por qué SPSS?}

SPSS ofrece una serie de beneficios que lo convierten en una opción
preferida para analistas y investigadores:

\begin{itemize}
\item
  \textbf{Interfaz intuitiva}: La interfaz permite preparar y analizar
  datos fácilmente, utilizando funcionalidades de arrastrar y soltar que
  eliminan la necesidad de escribir código.
\item
  \textbf{Gestión de datos simplificada}: Integra la gestión de datos
  con el análisis estadístico, facilitando la importación, limpieza y
  manipulación de datos.
\item
  \textbf{Análisis integral}: Realiza estadísticas descriptivas,
  análisis de regresión y visualiza patrones en los datos, todo dentro
  de una solución integral.
\item
  \textbf{Analítica predictiva avanzada}: Utiliza capacidades de
  modelado predictivo para pronosticar tendencias y resultados,
  mejorando la planificación y la investigación.
\item
  \textbf{Salida personalizable}: Adapta los resultados y reportes a tus
  necesidades específicas con cuadros, gráficos y tablas
  personalizables.
\item
  \textbf{Integración de código abierto}: Amplía la funcionalidad de
  SPSS utilizando R y Python a través de extensiones predefinidas o
  scripts personalizados.
\end{itemize}

\subsection*{Requisitos del sistema}\label{requisitos-del-sistema}
\addcontentsline{toc}{subsection}{Requisitos del sistema}

Para instalar SPSS, asegúrate de que tu sistema cumpla con los
siguientes requisitos:

\begin{itemize}
\item
  \textbf{Espacio en disco}: Mínimo de 2 GB de espacio libre en disco en
  cada servidor donde esté instalado SPSS.
\item
  \textbf{Memoria}: Mínimo de 4 GB de RAM.
\item
  \textbf{Procesador}: Procesadores de doble núcleo x64 (AMD 64 y EM64T)
  para Linux de 64 bits; procesador Power 8.
\end{itemize}

\subsection*{Beneficios de SPSS}\label{beneficios-de-spss}
\addcontentsline{toc}{subsection}{Beneficios de SPSS}

\begin{itemize}
\item
  \textbf{Interfaz de usuario intuitiva}: Facilita la preparación y
  análisis de datos sin necesidad de programar.
\item
  \textbf{Opciones de licencia flexibles}: Permite elegir entre
  diferentes opciones de compra, incluyendo licencias tradicionales y de
  suscripción.
\item
  \textbf{Aumente la productividad de la ciencia de datos}: Proporciona
  herramientas visuales accesibles tanto para programadores como para
  analistas no programadores.
\end{itemize}

\bookmarksetup{startatroot}

\chapter{Introducción a SPSS}\label{introducciuxf3n-a-spss}

\bookmarksetup{startatroot}

\chapter{Análisis Estadístico}\label{anuxe1lisis-estaduxedstico}

\bookmarksetup{startatroot}

\chapter{Estadística Correlacional}\label{estaduxedstica-correlacional}

\bookmarksetup{startatroot}

\chapter{Dominando los Gráficos}\label{dominando-los-gruxe1ficos}

\bookmarksetup{startatroot}

\chapter{Estadística Univariante}\label{estaduxedstica-univariante}

\bookmarksetup{startatroot}

\chapter{Estadística No
Paramétrica}\label{estaduxedstica-no-paramuxe9trica}

\bookmarksetup{startatroot}

\chapter{Análisis Multivariante}\label{anuxe1lisis-multivariante}

\bookmarksetup{startatroot}

\chapter{Estadística Bayesiana}\label{estaduxedstica-bayesiana}

\bookmarksetup{startatroot}

\chapter{Análisis de Series de
Tiempo}\label{anuxe1lisis-de-series-de-tiempo}

\bookmarksetup{startatroot}

\chapter{Redes Neuronales}\label{redes-neuronales}

\bookmarksetup{startatroot}

\chapter{Métodos de Clasificación}\label{muxe9todos-de-clasificaciuxf3n}

\bookmarksetup{startatroot}

\chapter{Diseño y Validación de
Encuestas}\label{diseuxf1o-y-validaciuxf3n-de-encuestas}

\bookmarksetup{startatroot}

\chapter{Programación}\label{programaciuxf3n}

\bookmarksetup{startatroot}

\chapter{Bibliografía}\label{bibliografuxeda}




\end{document}
