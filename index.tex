% Options for packages loaded elsewhere
\PassOptionsToPackage{unicode}{hyperref}
\PassOptionsToPackage{hyphens}{url}
\PassOptionsToPackage{dvipsnames,svgnames,x11names}{xcolor}
%
\documentclass[
  letterpaper,
  DIV=11,
  numbers=noendperiod]{scrreprt}

\usepackage{amsmath,amssymb}
\usepackage{iftex}
\ifPDFTeX
  \usepackage[T1]{fontenc}
  \usepackage[utf8]{inputenc}
  \usepackage{textcomp} % provide euro and other symbols
\else % if luatex or xetex
  \usepackage{unicode-math}
  \defaultfontfeatures{Scale=MatchLowercase}
  \defaultfontfeatures[\rmfamily]{Ligatures=TeX,Scale=1}
\fi
\usepackage{lmodern}
\ifPDFTeX\else  
    % xetex/luatex font selection
\fi
% Use upquote if available, for straight quotes in verbatim environments
\IfFileExists{upquote.sty}{\usepackage{upquote}}{}
\IfFileExists{microtype.sty}{% use microtype if available
  \usepackage[]{microtype}
  \UseMicrotypeSet[protrusion]{basicmath} % disable protrusion for tt fonts
}{}
\makeatletter
\@ifundefined{KOMAClassName}{% if non-KOMA class
  \IfFileExists{parskip.sty}{%
    \usepackage{parskip}
  }{% else
    \setlength{\parindent}{0pt}
    \setlength{\parskip}{6pt plus 2pt minus 1pt}}
}{% if KOMA class
  \KOMAoptions{parskip=half}}
\makeatother
\usepackage{xcolor}
\setlength{\emergencystretch}{3em} % prevent overfull lines
\setcounter{secnumdepth}{5}
% Make \paragraph and \subparagraph free-standing
\makeatletter
\ifx\paragraph\undefined\else
  \let\oldparagraph\paragraph
  \renewcommand{\paragraph}{
    \@ifstar
      \xxxParagraphStar
      \xxxParagraphNoStar
  }
  \newcommand{\xxxParagraphStar}[1]{\oldparagraph*{#1}\mbox{}}
  \newcommand{\xxxParagraphNoStar}[1]{\oldparagraph{#1}\mbox{}}
\fi
\ifx\subparagraph\undefined\else
  \let\oldsubparagraph\subparagraph
  \renewcommand{\subparagraph}{
    \@ifstar
      \xxxSubParagraphStar
      \xxxSubParagraphNoStar
  }
  \newcommand{\xxxSubParagraphStar}[1]{\oldsubparagraph*{#1}\mbox{}}
  \newcommand{\xxxSubParagraphNoStar}[1]{\oldsubparagraph{#1}\mbox{}}
\fi
\makeatother


\providecommand{\tightlist}{%
  \setlength{\itemsep}{0pt}\setlength{\parskip}{0pt}}\usepackage{longtable,booktabs,array}
\usepackage{calc} % for calculating minipage widths
% Correct order of tables after \paragraph or \subparagraph
\usepackage{etoolbox}
\makeatletter
\patchcmd\longtable{\par}{\if@noskipsec\mbox{}\fi\par}{}{}
\makeatother
% Allow footnotes in longtable head/foot
\IfFileExists{footnotehyper.sty}{\usepackage{footnotehyper}}{\usepackage{footnote}}
\makesavenoteenv{longtable}
\usepackage{graphicx}
\makeatletter
\def\maxwidth{\ifdim\Gin@nat@width>\linewidth\linewidth\else\Gin@nat@width\fi}
\def\maxheight{\ifdim\Gin@nat@height>\textheight\textheight\else\Gin@nat@height\fi}
\makeatother
% Scale images if necessary, so that they will not overflow the page
% margins by default, and it is still possible to overwrite the defaults
% using explicit options in \includegraphics[width, height, ...]{}
\setkeys{Gin}{width=\maxwidth,height=\maxheight,keepaspectratio}
% Set default figure placement to htbp
\makeatletter
\def\fps@figure{htbp}
\makeatother

\KOMAoption{captions}{tableheading}
\makeatletter
\@ifpackageloaded{bookmark}{}{\usepackage{bookmark}}
\makeatother
\makeatletter
\@ifpackageloaded{caption}{}{\usepackage{caption}}
\AtBeginDocument{%
\ifdefined\contentsname
  \renewcommand*\contentsname{Table of contents}
\else
  \newcommand\contentsname{Table of contents}
\fi
\ifdefined\listfigurename
  \renewcommand*\listfigurename{List of Figures}
\else
  \newcommand\listfigurename{List of Figures}
\fi
\ifdefined\listtablename
  \renewcommand*\listtablename{List of Tables}
\else
  \newcommand\listtablename{List of Tables}
\fi
\ifdefined\figurename
  \renewcommand*\figurename{Figure}
\else
  \newcommand\figurename{Figure}
\fi
\ifdefined\tablename
  \renewcommand*\tablename{Table}
\else
  \newcommand\tablename{Table}
\fi
}
\@ifpackageloaded{float}{}{\usepackage{float}}
\floatstyle{ruled}
\@ifundefined{c@chapter}{\newfloat{codelisting}{h}{lop}}{\newfloat{codelisting}{h}{lop}[chapter]}
\floatname{codelisting}{Listing}
\newcommand*\listoflistings{\listof{codelisting}{List of Listings}}
\makeatother
\makeatletter
\makeatother
\makeatletter
\@ifpackageloaded{caption}{}{\usepackage{caption}}
\@ifpackageloaded{subcaption}{}{\usepackage{subcaption}}
\makeatother

\ifLuaTeX
  \usepackage{selnolig}  % disable illegal ligatures
\fi
\usepackage{bookmark}

\IfFileExists{xurl.sty}{\usepackage{xurl}}{} % add URL line breaks if available
\urlstyle{same} % disable monospaced font for URLs
\hypersetup{
  pdftitle={Libro Análisis de Datos con SPSS},
  pdfauthor={Angel Miguel Alvarez Calicho},
  colorlinks=true,
  linkcolor={blue},
  filecolor={Maroon},
  citecolor={Blue},
  urlcolor={Blue},
  pdfcreator={LaTeX via pandoc}}


\title{Libro Análisis de Datos con SPSS}
\author{Angel Miguel Alvarez Calicho}
\date{2024-12-23}

\begin{document}
\maketitle

\renewcommand*\contentsname{Table of contents}
{
\hypersetup{linkcolor=}
\setcounter{tocdepth}{2}
\tableofcontents
}

\bookmarksetup{startatroot}

\chapter*{Bienvenida}\label{bienvenida}
\addcontentsline{toc}{chapter}{Bienvenida}

\markboth{Bienvenida}{Bienvenida}

Bienvenido este es un proyecto donde veremos las diferentes aplicaciones
de SPSS en el análisis estadístico, desde mi experiencia y estudios
realizados.

Mi nombre es Angel Miguel Alvarez Calicho soy graduado en la
Licenciatura en Ingeniería Comercial, con una Maestría en Estadística
Aplicada al Gestión Empresarial.

\bookmarksetup{startatroot}

\chapter{Introducción a SPSS}\label{introducciuxf3n-a-spss}

\bookmarksetup{startatroot}

\chapter{Análisis Estadístico con
SPSS}\label{anuxe1lisis-estaduxedstico-con-spss}

\bookmarksetup{startatroot}

\chapter{Estadística Correlacional con
SPSS}\label{estaduxedstica-correlacional-con-spss}

\bookmarksetup{startatroot}

\chapter{Dominando los Gráficos con
SPSS}\label{dominando-los-gruxe1ficos-con-spss}

\bookmarksetup{startatroot}

\chapter{Estadística Univariante con
SPSS}\label{estaduxedstica-univariante-con-spss}

\bookmarksetup{startatroot}

\chapter{Estadística No Paramétrica con
SPSS}\label{estaduxedstica-no-paramuxe9trica-con-spss}

\bookmarksetup{startatroot}

\chapter{Análisis Multivariante con
SPSS}\label{anuxe1lisis-multivariante-con-spss}

\bookmarksetup{startatroot}

\chapter{Estadística Bayesiana con
SPSS}\label{estaduxedstica-bayesiana-con-spss}

\bookmarksetup{startatroot}

\chapter{Análisis de Series de Tiempo con
SPSS}\label{anuxe1lisis-de-series-de-tiempo-con-spss}

\bookmarksetup{startatroot}

\chapter{Redes Neuronales con SPSS}\label{redes-neuronales-con-spss}

\bookmarksetup{startatroot}

\chapter{Métodos de Clasificación con
SPSS}\label{muxe9todos-de-clasificaciuxf3n-con-spss}

\bookmarksetup{startatroot}

\chapter{Diseño y Validación de Encuestas con
SPSS}\label{diseuxf1o-y-validaciuxf3n-de-encuestas-con-spss}

\bookmarksetup{startatroot}

\chapter{Programación con SPSS}\label{programaciuxf3n-con-spss}

\bookmarksetup{startatroot}

\chapter{Bibliografía}\label{bibliografuxeda}




\end{document}
